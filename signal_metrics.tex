\subsection{Preprocessing and signal metrics}

In this step, we preprocess synthetic and observation signals. The preprocessing involves base-line correction, tapering, and low pass filtering the signals. According to the previous studies, $Q$ is considered frequency independent up to \fmaxeq{1}. As a result, we lowpass filter the signals at \feq{1} corner frequency. We use $4^{th}$ order non-causal zero-phased Butterworth filter. $Q$ studies commonly use peak ground velocity or peak amplitude of S wave arrivals as an indicator of energy loss during the wave propagation \citep[e.g., see ][]{Olsen_2003_BSSA}. Unless in an entirely homogenous domain, it is not easy to pick the peak ground velocity for S wave arrival. In a complex geological structure, surface wave, direct S wave, and reflected/refracted body waves from different layers and interfaces are mixed. In that case, the energy is already dissipated in the wave propagation process, and the peak ground velocity is not very sensitive to the $Q$ parameters. Also picking the actual peak value of S wave arrivals is exceptionally prone to error, and it is not straightforward to separate body wave and surface wave windows in complicated geological regions \citep[e.g., see][]{Bowden_2017_GRL}. Moreover, our ideal case experiments prove that using only peak ground velocity will not necessarily provide better results even if we able to accurately pick the peak ground velocity for direct S wave. We will have more discussion on this issue in the result section. Therefore, in this study in addition to peak ground velocity (PGV) we use three other metrics including Peak ground acceleration (PGA), acceleration response spectra (SA) at the highest frequency(\feq{1}), and area under velocity signal envelop (Venv). These metrics are four metrics that Khoshenivs and Taborda(2018) recommended as the most sensitive metrics among others. We computed the signal envelope using the Hilbert transform. Arguably it is a similar proxy to the total energy of the signal. 












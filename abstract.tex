%
We present a customized solution approach to study attenuation models through combining machine learning, ground motion simulation, and optimization methods. The accurate solution of wave propagation problems requires the appropriate representation of energy losses due to internal friction in geomaterials. These losses are important because their mischaracterization may lead to the over- or under-estimation of the amplification and duration of seismic waves in regions with high dissipative properties. Recent studies show that synthetics from physics-based simulations tend to attenuate with distance at different rates than observations, thus suggesting that current approaches to modeling attenuation need to be revised. In physic-based ground-motion simulation, the attenuation of seismic waves is typically treated by means of viscoelastic models. Internally, the properties used for these models are set based on the material's quality factor Q. The value of Q for shear waves, Qs, for instance, is usually defined based on rules that depend on the value of the shear wave velocity, Vs.  Typical Qs-Vs relationships are (piecewise) linear or polynomial functions. Several Qs-Vs relationships exist in the literature. There is, however, no consensus about the most appropriate one. In addition, other studies suggest that these relationships vary for P waves, and are dependent of depth. 
We propose new methodology based on  combining machine learning (artificial neural network) techniques and optimization methods to narrow and customize the search domain for each individual stations in studying the Q factor parameters in physics-based ground motion simulation. We successfully tested the proposed method in homogeneous, layered, and actual heterogenous domains. The proposed method shows how emerging computational resources and machine learning techniques can improve scientific research. Working with observational data, due to noise, potential rotation of stations during earthquake, mistakes in preprocessing data and many more other factors, in many cases, make the search process very difficult. The proposed method, gives the idea of actual search area, before start woking with observations. We tested the proposed model based on $Mw ~5.4$ 2008 Chino Hills earthquake's recorded data and present the optimal Q model. 




%
We present a surrogate-based approach for analysis and optimization of parameters that are used in attenuation models in physics-based earthquake ground-motion simulation. The accurate solution of wave propagation problems requires the appropriate representation of energy losses due to internal friction in geomaterials. These losses are important because their mischaracterization may lead to the over- or under-estimation of the amplification and duration of seismic waves in regions with high dissipative properties. Recent studies show that synthetics from physics-based simulations tend to attenuate with distance at different rates than observations, thus suggesting that current approaches to modeling attenuation need to be revised. In physics-based earthquake ground motion simulation, the attenuation of seismic waves is typically treated by means of viscoelastic models. Internally, the properties used for these models are set based on the material's quality factor, $Q$. The value of $Q$ for shear waves, \qs{}, for instance, is usually defined based on rules that depend on the value of the shear wave velocity, \vs{}.  Typical \qsvs{} relationships are (piecewise) linear or polynomial functions. Several  \qsvs{} relationships exist in the literature. There is, however, no consensus about the most appropriate one.  Well-recorded earthquake data can be used in an optimization process to calibrate the parameters of attenuation model. Optimization process needs to run GMS hundreds of thousands of times, whereas generating accurate ground motion simulation models are computationally expensive and time-consuming. Therefore, based on numerous simulation results, we develop surrogates (or meta-models) to estimate the target values based on arbitrary \qsvs{} relationship input parameters. The surrogates are developed using artificial neural networks; and are used as the main input of cost function in an evolutionary algorithm.  For each station, we optimize the parameters to fit the synthetic signal metrics with observation. This study shows that using a sequence of optimization processes based on synthetic and observation data (as target values) can improve the confidence level on the final \qsvs{} relationship parameters. We present applications of the proposed method on homogeneous and layered domains. We discuss the application of the method on several recorded data of  $Mw ~5.4$ 2008 Chino Hills earthquake as a heterogeneous domain. This study shows that using other signal metrics can facilitate the optimization process of finding accurate solutions for $Q$ studies.




%We present a customized solution approach to study attenuation models through combining machine learning, ground motion simulation, and optimization methods. The accurate solution of wave propagation problems requires the appropriate representation of energy losses due to internal friction in geomaterials. These losses are important because their mischaracterization may lead to the over- or under-estimation of the amplification and duration of seismic waves in regions with high dissipative properties. Recent studies show that synthetics from physics-based simulations tend to attenuate with distance at different rates than observations, thus suggesting that current approaches to modeling attenuation need to be revised. In physic-based ground-motion simulation, the attenuation of seismic waves is typically treated by means of viscoelastic models. Internally, the properties used for these models are set based on the material's quality factor Q. The value of Q for shear waves, Qs, for instance, is usually defined based on rules that depend on the value of the shear wave velocity, Vs.  Typical Qs-Vs relationships are (piecewise) linear or polynomial functions. Several Qs-Vs relationships exist in the literature. There is, however, no consensus about the most appropriate one. We propose new methodology based on  combining machine learning (artificial neural network) techniques and optimization methods to narrow and customize the search domain for each station in studying the Q factor parameters in physics-based ground motion simulation. First we train a neural network to predict the requested signal's parameters, using the trained network as a cost function in optimization process, we optimized the parameters for recorded data and compute several potential Q parameters. We use the mean Q as an actual target value and repeat the optimization process. This step provides the dominant shear wave velocity ranges. We return back to the original optimization results and choose the Q data that belong to dominant shear wave velocity range. We present successful applications of the proposed method on homogeneous, layered, and actual heterogenous domains. The proposed method shows how integration of emerging computational resources with machine learning techniques can improve anelastic attenuation models' optimization processes. Analysis of observational data, due to noise, potential rotation of stations during earthquake, mistakes in preprocessing data and many more other factors, mostly leave a user and optimization algorithm with a very large search domain. The proposed method, on the other hand,  provides potential search domain which is tailored with observational data. We applied the proposed model on several recorded data of  $Mw ~5.4$ 2008 Chino Hills earthquake and discuss the results. 




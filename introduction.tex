
\section{Introduction}

The accurate solution of wave propagation problems requires the appropriate representation of energy losses due to internal friction in geomaterials. These losses are important because their mischaracterization may lead to the over- or under-estimation of the amplification and duration of seismic waves in regions with high dissipative properties. The amplitude of seismic waves decreases as the distance from source increases. In the absence of large deformations (i.e., nonlinearities), this is due to Geometric spreading, intrinsic attenuation, and scattering. Geometric spreading and scattering (as a result of heterogeneous velocity model) are inherently considered in $3D$ physics-based ground motion simulation. Intrinsic attenuation, on the other hand, is a result of irreversible changes in the crystal defect structures of the medium and should be considered in synthetic models. These media are called anelastic, and the configuration of material particles is to some extent dependent on the history of applied stress \citep{Aki_2002_Book}. Intrinsic attenuation or internal friction is studied by means of quality factor ($Q$), which is inversely related to the strength of the attenuation. From the definition of $Q$ (fractional energy loss per cycle), it is concluded that the rate of attenuation increases with frequency. Therefore, $Q$ is considered as a frequency-depended parameter \citep[e.g., see][ and references therein.]{Adams_1998_GJI, Mousavi_2014_JGR, Sedaghati_2015_BSSA, Nazemi_2017_TP} Specifically at \fgeq{1} \citep[e.g., see ][]{Withers_2015_BSSA}. High-frequency $Q$ can be dependent on numerous factors. \citet{Hauksson_2006_JGR} showed that $Q$ could be different on a regional scale due to local impedance contrasts, chemical composition, crack structure, grain boundary movement, crustal fluids, and, to a lesser extent, temperature variations within the brittle seismogenic crust. At lower frequencies (\fleq{1}), given the wavelengths and minimum velocities of physics-based ground motion simulation, $Q$ is considered as a function of shear wave velocity (\vs{}) and its frequency and depth dependency has been ignored.\\
In physics-based earthquake ground-motion simulation, the attenuation of seismic waves is typically treated by means of viscoelastic models \citep[e.g.,][]{Day_2001_BSSA, Graves_2003_BSSA, Kaser_2007_GJI, Bielak_2011_G}. These models are made of mechanisms that employ numerical techniques to account for the dissipative and transient characteristics of the ground motion. Internally, the properties of a given attenuation model are set based on the material's $Q$. Community velocity models ($CVMs$) used in simulations do not generally provide values of $Q$. Instead, the most common approach is to define the quality factors associated with $P$ and $S$ waves, \qp{} and \qs{}, based on the values of the seismic velocities \vp{} and \vs{} obtained from any particular $CVM$.\\
Several  \qsvs{} relationships exist in the literature. There is, however, no consensus about the most appropriate one. To the best of our knowledge, none of these studies have undertaken a comprehensive search for optimal $Q$ model. For example, \citet{Olsen_2003_BSSA} constructed several simple distributions of  \qs{} and \qp{} and identified those that provide the best fit between simulated and recorded $0-0.5~Hz$ peak velocities for the 1994 Northridge earthquake. Although their suggested models significantly improved the goodness-of-fit between synthetic and observation, the lack of consensus among different studies suggests the idea that a comprehensive search for optimal $Q$ should be considered. In this paper, we propose to use a parametric \qsvs{} relationship and we present a surrogate-based optimization method to better constraining the parameters of models used to represent the effects of intrinsic attenuation. We set up a genetic algorithm (GA) optimization process as an evolutionary algorithm to search for the best \qsvs{} relationship parameters by minimizing a cost function.  Cost functions, in simple words, are the absolute difference between proposed parameters solution and target values. Optimization process needs to run ground motion simulation hundreds of thousands of times, whereas generating accurate ground motion simulation models are computationally expensive and time-consuming. Therefore, based on numerous simulation results, we develop surrogates (or meta-models) to estimate the target values based on arbitrary \qsvs{} relationship input parameters. Although previous studies use peak ground velocity (PGV) as a metric for estimating energy loss \citep[e.g., see ][{\color{red} needs more references.}]{Olsen_2003_BSSA}, our study shows using other metrics can increase the uniqueness of signals and consequently optimization process. Therefore, the surrogates, which are developed using artificial neural networks, trained to estimate PGV, peak ground acceleration (PGA), acceleration response spectra (SA), and area under velocity signal envelop (Venp). Complicated source model, complex geological features, and stations distance from source make the study process unique for each station. In this study, we address this issue through our proposed method. \\
In summary, through a sequence of an optimization process, we locate the most effective shear wave velocity range that a station and optimization process, most probably, are capable of estimating the \qsvs{}, accurately.  relationship values. Therefore, from each station, we only choose those data, if there is any, which we have higher confidence in its accuracy. We test the proposed method on homogenous, layered, and heterogeneous domains.  We discuss the application of the method on a comparison of simulations for $Mw~5.4$ 2008 Chino Hills earthquake (\vsmin$=350~$m/s and \fmax$=1$~Hz) as a heterogeneous domain and compare the results with previous studies. Application of the method on different domains shows that it is robust, and based on the simulation model, station location, and metrics that are used, it can confidently locate the $Q$ values. 






























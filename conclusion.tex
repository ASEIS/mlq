\section{Conclusion}

We present a customized solution approach to study attenuation models through combining machine learning, ground motion simulation, and optimization process used in physics-based ground motion simulation. We train artificial neural networks and ensemble them though bagging approach as a sudo-simulator to estimate the signal parameters based on attenuation model inputs. For each station we estimate the approximate Q value through comparing the results with observation and then we understand at which velocity range the stations, model, and optimization process have the potential of providing accurate results. We test the proposed method on homogenous, layered and heterogenous simulation domains. We applied the proposed solution on several stations of 2008 $Mw$ 5.4 ChinoHills earthquake. The results are in agreement with previous studies. 

We recognize, however, that the proposed equation may not be a definitive one due to use of several stations of one earthquake. In the future follow-up study, it would be ideal to use more signal metrics, more stations and earthquake, as well as include source parameters as an input parameters in sudo-simulators. The procedural steps laid out here, nonetheless, remain valid.    

In summary, we can say that, in physics based ground motion simulation, artificial neural networks can be easily trained for estimating signal metrics. These networks can be used in optimization and uncertainty analysis studies. Example of using these networks with optimization algorithm can provide a dominant/effective shear wave velocity for each seismic station. We also showed that using only peak ground velocity as a signal metric for estimating Q factor parameters may not be enough and adding other parameters improve the results. A combination of machine learning algorithms, optimization process and ground motion simulation can customize the solutions for each individual station and this paper is a successful example of such an application on Q factor parameters studies. 




